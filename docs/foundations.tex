\documentclass[12pt]{article}
\usepackage[margin=1.0in]{geometry}
\usepackage{amsmath,amsthm,amssymb,amsfonts}
\usepackage{enumerate}

\title{Shortest Path Algorithm:\\
Foundations and Theoretical Concepts}
\author{
Mark Wesley Harris
}
\date{\today}

\begin{document}
\maketitle

\section{Introduction}\label{sec:intro}
This document breifly explains the theoretical concepts behind the
Shortest Path Algorithm developed in this repository. This algorithm attempts
to solve the Traveling Salesman Problem, with the input of a 2D graph of datapoints
represented on a Cartesian coordinate system.

\section{Input Requirements}\label{sec:req}
There are some requirements that must be met by the input to this algorithm.
These are outlined below:
\begin{enumerate}
\item Data must be represented on a 2D Cartesian plane
\item The weights between vertices must be the distance between them
(i.e. the weight between pionts $p:(x_1, y_1)$ and $q:(x_2, y_2)$ is $\sqrt((x_1 - x_2)^2 + (y_1 - y_2)^2)$)
\end{enumerate}

\section{Definitions}\label{sec:def}
\begin{itemize}
\item $(x,y)$: the definition of a point with coordinates $(x,y)$.
\item $<p,q>$: the definition of a vector from point $p$ to point $q$.
\item $[s_1, s_2, ..., s_k]$: the definition for a path through a known set of edges.
\item $n$: the number of points in the dataset.
\item $V$: the set of all vertices in the dataset.
Formally, $V = \{p_1, p_2, ..., p_n\}$
\item $E$: the set of all edges in the dataset.
The size of $E$ is $n \times n$.
\item $H$: the convex hull of the dataset.
Formally, $H = \{h_1, h_2, ..., h_k\}$, where $k \leq n - 1$.
Given any set of points,
there is only one convex hull.
\item $S$: the shortest path of a dataset as the set $S = \{s_1, s_2, ..., s_n\}$
of edges which are contained in the shortest path.
\item $E'$: the set of all edges in the dataset that are in $E$ but not $S$.
Formally, $E' = E - S$.
\item $W$: the set of edges produced by the construction phase of the algorithm.
Formally, $W = \{w_1, w_2, ..., w_m\}$, where $3 < m < n^2$.
\item $S' \equiv S$: the path $S'$ has the same perimeter as path $S$
(they may or may not have the same edges).
\end{itemize}

\section{Theoretical Concepts}\label{sec:theory}
My algorithm is based upon many theoretical concepts I have discovered through
research and experimentation. They are explained below in different sections.

\subsection{Properties of Shortest Paths}\label{subsec:props}
I begin by describing the properties I have found to be true for all shortest
paths. These properties have been used to create the foundations for my algorithm.
\begin{enumerate}
\item For any two edges $s_i,s_j \in S$ where $i \neq j$, $s_i$ and $s_j$ do not cross. 
\item For any two edges $s_i,s_j \in S$ where $i \neq j$, $s_i$ and $s_j$ do not overlap. 
\item No edge $s \in S$ intersects a point $p \in V$ where $s_1,s_2 \neq p$.
I.e. if an edge $e$ intersects a point $p$ that isn't an end point of $e$,
then $e \notin S$.
\item If we were to add an arbitrary point $p_{n+1}$ to $V$, $S$
will not change iff $p_{n+1}$ is a point on the line of any $s \in S$.
\end{enumerate}

\subsection{Invalidating a Segment}\label{subsec:invalid}
My algorithm is based upon the concept of invalidating as many segments of $E$
as possible, but keep all segments which could be in $S$. In order to do this,
I propose a method of construction that invalidates segments based upon the properties
above. Any segment which fails these properties cannot be in $S$.

\subsection*{Bijection Range Method}\label{subsec*:bijection-range-method}
...

\subsection*{Using Intersections Instead}\label{subsec*:intersection-method}
The Bijection Range Method works for determining ranges on a segment $w$
for which points $p,q \in V$ are closer to $w$ than $w_1$ or $w_2$.
However, since we are only interested
in invalidating the line with respect to this property, it is worth looking at
the possibility of using the intersection of any $e \in E$ instead.
\\\\
Claim: For each $p,q \in V$ where $p \neq w_1,w_2$ and $q \neq w_1,w_2$,
$p$ and $q$ have overlapping Bijection Ranges
if the line passing through them intersects $w$.

\begin{proof}
The proof is shown below:
\begin{enumerate}[(1)]
\item Given that $p$ and $q$ have overlapping Bijection Ranges,
it is evident that there are 2 intersection points formed by
the lower and upper bounds of the intersection of ranges.
Let these points be $w_l$ and $w_u$.
\item By Theorem \#.\#, these points form a quadralateral.
Let the quadralateral $p,w_l,q,w_u$ be denoted as $A$.
\item Since $A$ is a quadralateral,
a line segment can be drawn between $p$ and $q$ that lies within
the inner region formed by $A$.
Let that line be denoted as $e$.
\item Since $w_l$ and $w_u$ both intersect $w$, and $e$ is a continuous line
that lies between $w_l$ and $w_u$, then $e$ must also intersect $w$.
\end{enumerate}
\end{proof}

It is not sufficient, however, to claim that if a line $e \in E$ intersects $w$
that they then have overlapping Bijection Ranges.
Points $p$ and $q$ of line $e$ must also be within the circular area created by
revolving $w$ about it's midpoint.

\subsection{Construction Phase}\label{subsec:construct}
The goal of the construction phase of the algorithm is to generate $W$, i.e. a set of
edges $W = \{w_1, w_2, ..., w_m\}$ such that $W \subseteq E$ and $S \subseteq W$.
These edges must hold the properties of shortest paths, discussed in section
\ref{subsec:props}.
They must also give us some kind of meaningful information, so that we are able to
construct a path $S'$ from them with the claim that $S' \equiv S$.
The construction phase is broken up into steps below:
\begin{enumerate}
\item Choose an edge $e$ from $E$ with the points $e_1,e_2$.
We will test this edge to see if it is possible that $e$ is part of a shortest path,
and subsequently that $e \in W$.
\\\\
Take each edge $e' \in E$ with points $p,q$ such that $p$ and $q$ straddle
$e$ and where $p$ is on the right side
of $e$ and $q$ is on the left side of $e$.
We perform the following tests to see if $e'$ invalidates $e \in W$:
\begin{enumerate}
\item Let $V_1$ be the vector $<p,e_1>$.
\item Let $V_2$ be the vector $<p,e_2>$.
\item Find the bijection range of equality between $V_1,V_2$ and $e$,
represented as $r[lower, upper]$, as follows:
\begin{enumerate}
\item Let vector $B$ start from the bijection of $V_1$
in the direction of $e$ with length $|V_2|$.
\item Let the upper bound of $r$ be the intersection of $B$ and $e$.
\item Let $B$ start from the bijection of $V_2$
in the direction of $e$ with length $|V_1|$.
\item Let the lower bound of $r$ be the intersection of $B$ and $e$.
\end{enumerate}
\item Repeat steps (a) through (c) with $q$ instead of $p$.
\item If any of the intersections failed, $e'$ does not invalidate $e$,
so choose the next $e'$ and repeat from the beginning.
\item If all intersections suceeded, find the overall intersection area between
both $r$ ranges generated from steps (a) through (c) using $p$ and $q$.
\item If this intersection area is greater than the threshold, then $e$ is proven
invalid by $e'$. Choose the next $e$ and repeat from the beginning.
\end{enumerate}
\item If no $e'$ invalidated $e$, then $e \in W$.
\end{enumerate}

\subsection{Expectations for $W$}\label{subsec:exp_w}
Based upon the properites of shortest paths, $W$ should have the following expectations:
\begin{enumerate}
\item $H$ will always be included, since there are no paths which could prove to
invalidate any $h \in H$.
\item Any two edges $w_i,w_j \in W$ where $i \neq j$ can cross each other.
However any path $S'$ can only contain one of these edges.
\item If an arbitrary point $p_{n+1}$ was added on an edge $w \in W$,
$w$ would not be intersected by an edge $e \in E$ where $e \neq w$.
\\\\
Derived from the construction phase of the algorithm,
the sum of the distances between points
$w_1$ and $w_2$ and the point $p_{n+1}$
is smaller than the sum of the distances between points
$e_1$ and $e_2$ and the point $p_{n+1}$ for all $e \in E$ such that $e \neq w$.
Thus, if it were true that $p_{n+1} \in V$, the only path in $W$ passing
through $p_{n + 1}$ would be the path $[w_1,p_{n+1},w_2]$.
All other paths have been proven to be too long.
Thus $[e_1,p_{n+1},e_2] \equiv [w_1,p_{n+1},w_2] \Rightarrow w \equiv e$
and $e$ cannot cross $w$.
\item A path $S'$ can be made from $W$ which goes through all vertices in $V$ only once
and starts and ends on the same point.
If $W$ has multiple paths that fit these requirements, then all such paths
$\{S'_1, S'_2, ... S'_k\}$ should be collected.
The shortest path in this set must be equivalent to $S$.
\end{enumerate}

\end{document}
